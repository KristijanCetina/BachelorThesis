\documentclass[11pt,a4paper]{report}
\usepackage[utf8]{inputenc}
\usepackage[croatian]{babel}
\usepackage{amsmath, amsfonts, amssymb}
\usepackage{graphicx}
\usepackage{fancyhdr}
\usepackage{color}
\usepackage {tikz}
\usepackage{pgfplots}
\usetikzlibrary {positioning}
\usepackage{tocloft}
\usepackage[hidelinks]{hyperref}
\usepackage[section]{placeins}
\usepackage[final]{pdfpages}
\bibliographystyle{abbrv}%ieeetr, abbrv
\renewcommand{\cftsecleader}{\cftdotfill{\cftdotsep}}
\addto{\captionscroatian}{\renewcommand{\bibname}{Literatura}}

\usepackage[blocks]{authblk}% The option is for block layout

\newcommand{\kolegij}{}
\newcommand{\naslovRada}{Arduino datalogger \\ {\large Završni rad}} 
\newcommand{\mailFriendlynaslovRada}{BachelorThesis}

\author{
Kristijan Cetina \\{\small JMBAG: 2424011721} \\ {\href{mailto:kcetina@politehnika-pula.hr?subject=\mailFriendlynaslovRada}{{\footnotesize kcetina@politehnika-pula.hr}}}} 
%\author[1]{Kristijan Cetina}

%\affil[1]{\href{mailto:kcetina@politehnika-pula.hr?subject=\mailFriendlynaslovRada}{kcetina@politehnika-pula.hr} JMBAG: 2424011721}

%\title{\kolegij \\ \naslovRada}
\title{\naslovRada}
\date{Pula, \today}

\begin{document}
\pgfplotsset{width=\textwidth,compat=newest}

\begin{titlepage}
\clearpage
\begin{center}
\begin{Huge}
POLITEHNIKA PULA\\
\end{Huge}
\begin{LARGE}
Visoka tehničko-poslovna škola s p.j.\\
Stručni studij politehnike\\
\end{LARGE}
\end{center}
\vspace{3cm}
{\let\newpage\relax\maketitle}
\thispagestyle{empty}
\vfill

\end{titlepage}
\newpage
\vspace*{\fill}
\begin{flushright}
\textit{Posveta}
\end{flushright}
\vspace*{\fill}
\newpage
\section*{Zahvala}
Zahvala svima koji zaslužuju
\newpage
\section*{Izjava o samostalnosti izrade završnog rada}
Izjavljujem da sam završni rad na temu „Arduino datalogger“ samostalno izradio uz pomoć mentorice Sanje Grbac Babić mag. računarstva, koristeći navedenu stručnu literaturu i znanje stečeno tijekom studiranja. Završni rad je pisan u duhu hrvatskog jezika.
\vspace{3cm}
\begin{flushright}
Student: Kristijan Cetina\\
\vspace{15mm}
--------------------------------------------------------------
\end{flushright}
\newpage
\begin{abstract}

U ovom radu predstavljam

\vspace{3cm}

\begin{tabbing}
\hspace{50pt}\=\kill
 Kolegij: \> Elektronike\\
 Mentorica: \> Sanja Grbac Babić, mag. računarstva, v.predavač
\end{tabbing} 

\end{abstract}
\newpage
\tableofcontents
\listoftables	%ako ih ima puno prebaci na kraj dokumenta
\listoffigures	%ako ih ima puno prebaci na kraj dokumenta

%Ovdje kreće rad ili može se koristiti
%\input{filename} (da nastavi pisati kako da je copy/paste) ili
%\include{filename} (da ubaci na novu stranicu. Ok za nova poglavlja i sl)
\chapter{Opis zadatka i ograničenja}\label{OpisIOgranicenja}
\section{Uvod}

\newpage
\nocite{*}
\addcontentsline{toc}{chapter}{Literatura}
\bibliography{literatura}

\newpage
\appendix

\end{document}