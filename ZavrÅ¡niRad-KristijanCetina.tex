\documentclass[11pt,a4paper,openright]{report}
\usepackage[utf8]{inputenc}
\usepackage[english, croatian]{babel}
\usepackage{amsmath, amsfonts, amssymb}
\usepackage{graphicx}
\usepackage{fancyhdr}
\usepackage{color}
\usepackage {tikz}
\usepackage{pgfplots}
\usetikzlibrary {positioning}
\usepackage{tocloft}
\usepackage[hidelinks]{hyperref}
\usepackage[section]{placeins}
\usepackage[final]{pdfpages}
\bibliographystyle{abbrv}%ieeetr, abbrv
\renewcommand{\cftsecleader}{\cftdotfill{\cftdotsep}}
\addto{\captionscroatian}{\renewcommand{\bibname}{Literatura}}
\setcounter{chapter}{-1}

\pagenumbering{Roman}

%\usepackage[blocks]{authblk}% The option is for block layout

\newcommand{\kolegij}{Završni rad}
\newcommand{\naslovRada}{GPS datalogger \\ {\large Završni rad}} 
\newcommand{\mailFriendlynaslovRada}{BachelorThesis}

\author{
Kristijan Cetina \\{\small JMBAG: 2424011721} \\ {\href{mailto:kcetina@politehnika-pula.hr?subject=\mailFriendlynaslovRada}{{\footnotesize kcetina@politehnika-pula.hr}}}} 
%\author[1]{Kristijan Cetina}

%\affil[1]{\href{mailto:kcetina@politehnika-pula.hr?subject=\mailFriendlynaslovRada}{kcetina@politehnika-pula.hr} JMBAG: 2424011721}

%\title{\kolegij \\ \naslovRada}
\title{\naslovRada}
\date{Pula, \today}

\begin{document}
\pgfplotsset{width=\textwidth,compat=newest}

\begin{titlepage}
\clearpage
\begin{center}
\begin{Huge}
POLITEHNIKA PULA\\
\end{Huge}
\begin{LARGE}
Visoka tehničko-poslovna škola s p.j.\\
Stručni studij politehnike\\
\end{LARGE}
\end{center}
\vspace{3cm}
{\let\newpage\relax\maketitle}
\thispagestyle{empty}
%\vfill
\begin{abstract}

U ovom radu predstavljam
Lorem ipsum dolor sit amet, consectetur adipiscing elit. Vestibulum pretium libero non odio tincidunt semper. Vivamus sollicitudin egestas mattis. Sed vitae risus vel ex tincidunt molestie nec vel leo. Vestibulum ante ipsum primis in faucibus orci luctus et ultrices posuere cubilia Curae; Maecenas quis massa tincidunt, faucibus magna non, fringilla sapien. In ullamcorper justo a scelerisque egestas. Ut maximus, elit a rutrum viverra, lectus sapien varius est, vel tempor neque mi et augue. Fusce ornare venenatis nunc nec feugiat. Proin a enim mauris. Mauris dignissim vulputate erat, vitae cursus risus elementum at. Cras luctus pharetra congue. Aliquam id est dictum, finibus ligula sed, tempus arcu. \\

\textbf{Ključne riječi:} \textit{rijec, dva, tri \ldots}

\vspace{3cm}

\begin{tabbing}
\hspace{50pt}\=\kill
 Kolegij: \> Elektronika\\
 Mentorica: \> Sanja Grbac Babić, mag. računarstva, v.predavač
\end{tabbing} 
\end{abstract}

\begin{otherlanguage}{english} 
\begin{abstract}
Abstract in English\\
Lorem ipsum dolor sit amet, consectetur adipiscing elit. Vestibulum pretium libero non odio tincidunt semper. Vivamus sollicitudin egestas mattis. Sed vitae risus vel ex tincidunt molestie nec vel leo. Vestibulum ante ipsum primis in faucibus orci luctus et ultrices posuere cubilia Curae; Maecenas quis massa tincidunt, faucibus magna non, fringilla sapien. In ullamcorper justo a scelerisque egestas. Ut maximus, elit a rutrum viverra, lectus sapien varius est, vel tempor neque mi et augue. Fusce ornare venenatis nunc nec feugiat. Proin a enim mauris. Mauris dignissim vulputate erat, vitae cursus risus elementum at. Cras luctus pharetra congue. Aliquam id est dictum, finibus ligula sed, tempus arcu. \\

\textbf{Keywords:} \textit{rijec, dva, tri \ldots}

\end{abstract}
\end{otherlanguage}

\end{titlepage}
\newpage

\vspace*{\fill}
\begin{flushright}
\textit{Posveta}
\end{flushright}
\vspace*{\fill}
\newpage

\section*{Zahvala}
Zahvala svima koji zaslužuju
\newpage

\section*{Izjava o samostalnosti izrade završnog rada}
Izjavljujem da sam završni rad na temu "GPS datalogger" samostalno izradio uz pomoć mentorice Sanje Grbac Babić mag. računarstva, koristeći navedenu stručnu literaturu i znanje stečeno tijekom studiranja. Završni rad je pisan u duhu hrvatskog jezika.
\vspace{\fill}
\begin{flushright}
Student: Kristijan Cetina\\
\vspace{15mm}
--------------------------------------------------------------
\end{flushright}
\newpage

\tableofcontents
\listoftables	%ako ih ima puno prebaci na kraj dokumenta
\listoffigures	%ako ih ima puno prebaci na kraj dokumenta
\newpage

\pagenumbering{arabic}
%Ovdje kreće rad ili može se koristiti
%\input{filename} (da nastavi pisati kako da je copy/paste) ili
%\include{filename} (da ubaci na novu stranicu. Ok za nova poglavlja i sl)

\chapter{Opis zadatka i ograničenja}\label{OpisIOgranicenja}
\section{Uvod}

\newpage
\nocite{*}
\addcontentsline{toc}{chapter}{Literatura}
\bibliography{literatura}

\newpage
\appendix

\end{document}