\documentclass[11pt,a4paper]{report}
\usepackage[utf8]{inputenc}
\usepackage[T1]{fontenc}
\usepackage[english, italian, croatian]{babel}
\usepackage{amsmath, amsfonts, amssymb}
\usepackage{graphicx}
\usepackage{fancyhdr}
\usepackage{color}
\usepackage {tikz}
\usepackage{pgfplots}
\usetikzlibrary {positioning}
\usepackage{tocloft}
\usepackage[hidelinks]{hyperref}
\usepackage[section]{placeins}
\usepackage[final]{pdfpages}
\bibliographystyle{ieeetr}%ieeetr, abbrv
\pgfplotsset{compat=1.15}
\usepackage{listings}
\usepackage{appendix}
\usepackage{lipsum}
%\usepackage{chngcntr}	%Continuous footnote numbering
%\counterwithout{footnote}{chapter} %Continuous footnote numbering

\renewcommand{\cftsecleader}{\cftdotfill{\cftdotsep}}
\addto{\captionscroatian}{\renewcommand{\bibname}{Literatura}}
%\setcounter{chapter}{-1} % zero-based numbering
\pagenumbering{Roman}

\newcommand{\kolegij}{Završni rad}
\newcommand{\naslovRada}{Izrada snimača podataka, obrada i vizualizacija prikupljenih podataka bazirana na principima slobodnog i otvorenog koda \\ {\large Završni rad}} 
\newcommand{\mailFriendlynaslovRada}{BachelorThesis - FOSS Datalogger}

\author{
Kristijan Cetina \\{\small JMBAG: 2424011721} \\ {\href{mailto:kristijan.cetina@gmail.com?subject=\mailFriendlynaslovRada}{{\footnotesize kcetina@iv.hr}}}} 
\title{\naslovRada}
\date{Pula, \today}

\begin{document}
\pgfplotsset{width=\textwidth,compat=newest}

\begin{titlepage}
\clearpage
\begin{center}
\begin{Large}
ISTARSKO VELEUČILIŠTE\\
UNIVERSIT\`{A} ISTRIANA DI SCIENZE APPLICATE\\
Stručni studij politehnike\end{Large}
\end{center}
\vspace{3cm}
{\let\newpage\relax\maketitle}
\thispagestyle{empty}

\begin{abstract}
Prikupljanje, obrada i vizualizacija podataka nalazi se na svakom koraku našeg života.
U ovom radu prikazan je postupak izrade snimača podataka koristeči Arduino platformu te obrada podataka koristeći iskljućivo alate otvorenog i slobodnog koda.
Hardwareska platforma trenutno koristi GPS prijemnik i temperaturni senzor te se u budućnosti planira povećanje broja senzora.
Podaci koji se skupljaju na uređaju se spremaju lokalno na memorijsku karticu te se kasnije analiziraju na računalu.
Za analizu je korišten programski jezik Python unutar Jupyter notebook okruženja uz pomoć Pandas, NumPy i Matplotlib biblioteka.
Izrađen alat za analizu radi na svim popularnim operacijskim sustavima danas u upotrebi.

U duhu otvorenog koda i dobre prakse znanstvene zajednice ovaj rad je napisan koristeći \LaTeX{} sustav te je kompletan rad \textit{hostan} na GitHub-u i slobodano dostupan svim zainteresiranim stranama koji žele nastaviti rad na ovoj temi.

\paragraph{Ključne riječi} \textit{Arduino, Python, Matplotlib, \textit{open source}, obrada podataka}
\vspace{3cm}

\begin{tabbing}
\hspace{60pt}\=\kill
 Kolegij: \> Elektronika\\
 Mentorica: \> Sanja Grbac Babić, mag. računarstva, v.predavač
\end{tabbing} 
\end{abstract}

\begin{otherlanguage}{english} 
\begin{abstract}
Data acquisition, processing and visualization is at every step of our lives.
This paper describes how to record data using an Arduino and analyse them using free and open source (FOSS) tools exclusively.
The hardware platform is currently using a GPS receiver and a temperature sensor with planned increase of the number of sensors in future.
The data collected on the devices is stored locally to a memory card and subsequently analyzed on a computer.
The Python programming language within the Jupyter notebook environment was used for analysis with the help of the Pandas, NumPy and Matplotlib libraries.
The created analysis tool works on all today popular operating systems.

In the spirit of the open source and good practice of the scientific community, this paper was written using the \LaTeX{} system and a complete work is hosted on GitHub and is freely available to all interested parties who wish to work on these topics.

\paragraph{Keywords:} \textit{Arduino, Python, Matplotlib, \textit{open source}, data analysis}
\end{abstract}
\end{otherlanguage}

\begin{otherlanguage}{italian} 
\begin{abstract}
La raccolta, l’elaborazione e la visualizzazione dei dati è presente in ogni aspetto della nostra
vita.
In questa tesi è documentata la creazione di uno strumento per la raccolta dati creato su
piattaforma Arduino, e \mbox{l’elaborazione} dei dati raccolti usando esclusivamente strumenti \textit{Open Source}.
La piattaforma hardware attuale usa un ricevitore GPS ed un sensore di temperatura mentre per future evoluzioni è in programma l’incremento del numero di sensori. 
I dati raccolti dallo strumento vengono immagazzinati sulla scheda di memoria per poi essere analizzati sul computer.
Per l’analisi dei dati viene usato il linguaggio di programmazione Python sull' ambiente di \mbox{programmazione} Jupyter notebook, con l’ausilio di biblioteche Pandas, \mbox{NumPy} e Matplotlib.
Lo strumento qui descritto è compatibile con tutti i sistemi operativi piu popolari oggi in uso.

Nello spirito della filosofia Open Source e della cultura scientifica questa tesi e stata scritta usando il sistema \LaTeX{} ed è disponibile liberamente su GitHub per tutti coloro che fossero interessati a continuare il lavoro fatto su questo tema.

\paragraph{Parole chiave:} \textit{Arduino, Python, Matplotlib, \textit{open source}, analisi dati}
\end{abstract}
\end{otherlanguage}
\end{titlepage}

\vspace*{\fill}
\begin{flushright}
\textit{Posveta}
\end{flushright}
\vspace*{\fill}
\chapter*{Zahvala}
Zahvaljujem svojoj mentorici Sanji Grbac Babić, mag. računarstva, višoj predavačici na izdvojenom vremenu i podršci, kako na izradi ovog rada, tako i tijekom cijelog studiranja na Politehnici.
Svojm smirenim pristupom uvelike je olakšala proces učenja i rada, bilo da se radilo o stručnim tehničkim pitanjima ili ostalim izazovima s kojima se studenti susreću.

Zahvaljujem se i svojim timskim kolegama, Stjepanu Grginu i Igoru Mrkiću, s kojima sam od samog početka sudjelovao na svim timskim zadatcima i njihovoj pomoći pri individualnom radu. 
Team \textit{One} rocks.

Zahvaljujem se i svim ostalim profesorima i djelatnicima Politehnike koja je tokom studija narasla i preimenovana u Istarsko Veleučilište - Universit\`{a} Istriana Di scienze applicate na nesebičnoj potpori kada je god to bilo potrebno.
Vi činite ovu ustanovu ono što ona je.

Naposljetku veliko hvala mojoj obitelji na potpori i razumijevanju tijekom mojeg ponovnog studiranja.
Draga mama, iako više nisi s nama znam da bi bila sretna.
Neizmjerno Ti hvala na svemu.

\vspace*{2cm}

\begin{quotation}
"A good scientist is a person with original ideas. A good engineer is a person who makes a design that works with as few original ideas as possible. There are no prima donnas in engineering" - Freeman Dyson
\end{quotation}
\chapter*{Izjava o samostalnosti izrade završnog rada}
Izjavljujem da sam završni rad na temu \textbf{\emph{Izrada snimača podataka, obrada i vizualizacija prikupljenih podataka bazirana na principima slobodnog i otvorenog koda}} samostalno izradio uz pomoć mentorice Sanje Grbac Babić mag. računarstva, koristeći navedenu stručnu literaturu i znanje stečeno tijekom studiranja.
Završni rad je pisan u duhu hrvatskog jezika.
\vspace{\fill}
\begin{flushright}
Student: Kristijan Cetina\\
\vspace{15mm}
--------------------------------------------------------------
\end{flushright}

\tableofcontents
%\listoftables	%ako ih ima puno prebaci na kraj dokumenta
\listoffigures	%ako ih ima puno prebaci na kraj dokumenta
\newpage

\pagenumbering{arabic}
\chapter{Uvod i opis zadatka}\label{OpisIOgranicenja}
Tema iz ovog rada nastala je iz autorove želje za proučavanjem tematike te kao gorljivi poklonik medote učenja kroz praktičan rad i primjenu stečenog znanja i iskustva na rješavanje realnog problema nastao je ovaj rad.
Dodatno, autorova želja je implementirati stečena znanja i razvijeni hardware i software u budućim projektima koje ima na umu.
Koristeči dostupna i otvorena znanja značajno je lakše razvijati potrebne projekte, a otvarajući rad ostalim zainteresiranom stranama omogućava se kolaboracija na usavršavanju projekta ili stvaranju baze za druge, kompeksnije projekte.

Arduino platforma je odabrana za korištenje u hardwareskom djelu rada jer ista predstavlja relativno jednostavan i povoljan pristup te uz veliki izbor dodatnih gotovih pločica za proširivanje mogućnosti (\textit{shields}) s otvorenom dokumentacijom logičan je izbor.

Na softwareskoj strani izbor je pao na programski jezik Python te dodatne biblioteke Matplotlib i NumPy koje su isto tako otvorenog koda i besplatne za korištenje i dovoljno jednostavne za učenje.
Dodajmo na to popularnost među akademskom i znanstvenom zajednicom isti se nameću kao logičan izbor.

\section{Opis i definicija problema}
Pri proučavanju bilo kojeg inženjerskog ili znanstvenog problema potrebno je prikupiti dovoljnu količinu podataka te iste analizirati i na reprezentativan način prikazati u pokušaju razumjevanja problema.
GPS je glogalno dostupan sustav putem kojeg se mogu prikupiti dovoljno precizni podaci za civilnu primjenu te je format podataka standardiziran te se zbog toga mogu koristiti i standardne metode analize podataka.
Ovisno o promatranom problemu potrebni su razni dodatni podaci koje pružaju razni senzori.
Temperatura je jedan o njih te je u ovom radu korišten klasični TMP36 temperaturni senzor i obrađeni podaci koji su skupljeni putem navedenog senzora.

\section{Cilj i svrha rada}
Cilj ovog rada je bio izraditi jednostavan snimač podataka (\textit{datalogger}) koji će spremati GPS podatke zajedno s podacima prikupljenim sa instaliranih senzora za kasniju analizu.
Izrađeni uređaj je namjenjen kao snimač podataka u kompleksnijem sklopu koji se može koristiti kada god postoji potreba za loggiranje podataka.
Uređaj je namjenjen da zadovolji široki spektar potreba koje se mogu javiti bilo u industriji npr. prilikom praćenja pošiljki ili pak prilikom skupljanja podataka u istraživačke svrhe kako bi se razumio širi problem.

Sklop je baziran na Arduino platformi koja omogućava lak razvoj prototipova uz široku dostupnost gotovih dodatnih modula (\textit{shields}) koji se jednostavno spajaju na bazno mikroračunalo.

Prikpljeni podaci se spremaju na SD karticu na uređaju u datoteku za kasniju obradu i analazu.
Prikupljeni podaci se uz pomoć programskog jezika Python i dodatnih modula za statističku i numeričku analizu kao što su Pandas i Matplotlib obrađuju kroz sučelje interaktivne bilježnice Jupyter Notebook.
Pristup obrade putem interaktivne bilježnice uz korištenje raznih tipova čelija kao što su \textit{Code Cells, Markdown Cells i Raw Cells} omogućava lakšu vizualizaciju i pregled samog rada koji je pogodan za kasnije dijeljenje svim zainteresiranim stranama koji žele pregledati ili nastaviti rad na analizi.

\section{Hipoteza rada}
Hipoteza ovog rada je da se primjenom pristupa otvorenog i slobodnog koda (računanog koda, metode obrade podataka, algoritama i shema hardwareskog sklopa) može izraditi funkcionalan sustav uz prihvatljive troškove i skromne resurse.

\section{Metode rada}
Tijekom izrade ovog rada korištene su različite znanstveno-istraživačke metode od kojih je svaka najprikladnija postavljenom izazovu, a one su:
\begin{itemize}
\item Istraživačka metoda – za stjecanje uvida u zadane okvire zadatka
\item Metoda logičke analize i sinteze – za prikupljanje podataka iz literature
\item Deskriptivna metoda – za izradu uvodnog i završnog djela projektnog zadatka
\item Eksperimentalna metoda - u potrazi za optimalnim rješenjima za zadani dio problema
\end{itemize}

\section{Struktura rada}
Struktura ovog rada podjeljena je u logičke cjeline.
Nakon uvoda i objašanjenja rada u poglavlju \ref{technologyStack} opisana je korišten pristup i primjenjene tehnologije kao i dano objašnjenje zašto je ista upotrebljena.

U poglavlju \ref{ch:Hardware} opisan je izrađen hardwareski elektronički sklop korišten za prikupljanje obrađenih podataka.
U dodatku \ref{ArduinoSource} priložen je izvorni kod koji se izvršava na Arduino mikroračunalu.

Poglavlje \ref{ch:Software} opisuje postupak izrade analize prikupljenih podataka.
Kompletna analiza se nalazi u dodatku \ref{AnalizaKretanjaVozila}

Kompletan Git repozitorij ovog rada javno je dostupan na \url{https://github.com/KristijanCetina/BachelorThesis}
\chapter{Opis korištenih tehnologija}\label{technologyStack}
\section{Slobodan i otvoreni kod}
Izraz otvoreni kod (\textit{open source}) odnosi se na nešto što ljudi mogu slobodno mijenjati i dijeliti jer je dizajn javno dostupan\cite{WhatIsOpenSource}.
Izraz je nastao u kontekstu razvoja računalnog softwarea dok se danas odnosi na pristup radu bio on software, hardware ili kakav drugi tip projekta.
Važno je napomenuti kako postoje razne licence pod kojima se objavljuju open source radovi, a u praksi se razlikuju u načinu na koji izmjenjeni i izvorni rad mora biti distribuiran svim ostalim zainteresiranim stranama.

Razlozi i prednosti primjene open source pristupa projektima su višestruke, a neki od njih su:
\begin{itemize}
\item Kontrola proizvoda
\item Učenje i trening
\item Sigurnost
\item Stabilnost
\end{itemize}

\paragraph{Kontrola proizvoda:}
Kada je izvorni kod i ostala dokumentacija nekog proizvoda otvorena, tada se može pogledati kako točno radi taj proizvod i na koji je način izgrađen.
Time svaki korisnik može imati kontrolu nad onime što koristi jer ne postoji koncept crne kutije (\textit{BlackBox concept}) što omogućava da uz dostupan kod i sheme popravi ili unaprijedi proizvod.
Zapitajmo se koliko puta smo se osobno susreli sa situacijom da smo zbog kvara nekog uređaja bili primorani posjetiti i platiti ovlaštenog servisera koji ima specijalni alat za dijagnostiku i popravke?

\paragraph{Učenje i trening:}
Uvidom u otvorenu dokumentaciju možemo vidjeti kako je neki stručnjak riješio određeni problem te se to rješenje u potpunosti ili modificirano može primjeniti na vlastiti.
Otvorena dokumentacija omogućava proučavanje rješenja određenih problema skraćujući vrijeme i pojeftinjuje razvoj novih proizvoda koji imaju slične zahtjeve.
Znanstvenici objavljuju svoja otkrića ne bili ih mogli koristiti.
Inženjeri u svakodnevnom radu ne izvode i dokazuju npr. Ohmov ili Newtonove zakone, već ih samo primjenjuju.

\paragraph{Sigurnost:}
Proučavanjem objavljene dokumentacije projekta drugi stručnjaci iz toga područja mogu uvidjeti neke propuste koje autori zbog kompleksnosti proizvoda ili drugih razloga nisu primjetili te dojaviti autorima pogrešku ne bili se ista mogla ispraviti. 
Neke pogreške mogu se pojaviti u iznimno malom broju slučaja ili kada se posloži veliki broj faktora te nije realno očekivati da se prilikom testiranja proizvoda simulira svaki mogući scenarij korištenja.
Zainteresirane strane mogu dodatno testirati proizvod u specifičnim uvjetima i na taj način otkriti inače skrivenu pogrešku u proizvodu čijim otklanjanjem proizvod postaje sigurnijim.

\paragraph{Stabilnost:}
Mnogi proizvodi koriste se za vrlo važne aspekte rada nekog većeg sustava te njihova zamjena iziskuje velike promjene i investicije, a ponekada nije niti moguća.
Korištenjem proizvoda i dokumentacije otvorenog koda omogućava se korištenje uz nastavak podrške kao i njegovo korištenje nakon eventualnog nestanka kompanije koja je napravila proizvod te isti više nije dobavljiv od proizvođača. 
Ako se koriste open source proizvodi moguće je, ukoliko se ukaže potreba, samostalno rekreirati proizvod.


\section{Arduino platforma}
Arduino je elektronička platforma otvorenog koda\footnote{\url{https://www.arduino.cc/en/Guide/Introduction}} temeljena na hardwareu i softwareu koji je lako za koristiti.
Arduino platforma obuhvaća mikrokontrolerske pločice temeljene na AVR arhitekturi s integriranim digitalnim, analognim ulazima i izlazima kao i PWM\footnote{Pulse Width Modulation - Pulsno širinska modulacija} izlazima.
Platforma omogućuje jednostavno spajanje dodatnih vanjskih uređaja poput raznih senzora, releja, serva i motora putem dodatnih upravljačkih modula te ostale elektroničkih i elektromehaničkih komponenti.
Sheme svih mikrokontrolera objavljene su pod Creative Commons\footnote{\url{https://creativecommons.org/}} licencom te javno dostupne svim zainteresiranim stranama.

Adruino pločice relativno su povoljne u usporedbi s ostalim platformama i kao takve omogućavaju pristupačnije učenje svim zainteresiranima.
Potrebno je ponekad malo spretnosti s lemilicom, iako se često mogu slagati moduli na prototipnoj pločici bez lemljenja sa izradom spojeva putem spojnih žica.

Jednostavno korisničko sučelje (IDE\footnote{Integrated development environment - Integrirano razvojno okruženje}) za izradu korisničkih programa (\textit{sketch}) je jednostavno za korištenje početnicima dok istovremeno iskusnim korisnicima omogućava izradu vrlo kompleksnih programa.
IDE je kompatibilan s većinom danas rasprostranjenih operacijskih sustava (GNU/Linux, MacOS i Windows).
Programski jezik za izradu programa je temeljen na C/C++ te omogućava daljnje proširivanje kroz C++ biblioteke ili korištenje AVR-C programskog jezika.

\section{Jupyter Notebook}
U ovom radu za obradu i prikazivanje podataka korišten je programski jezik Python\footnote{\url{https://www.python.org/}} uz dodatke NumPy\footnote{\url{https://numpy.org/}}, Pandas\footnote{\url{https://pandas.pydata.org/}} i Matplotlib\footnote{\url{https://matplotlib.org/}}.
NumPy i Pandas omogućuju lakšu manipulaciju podacima dok Matplotlib omogućuje izradu kvalitetnih grafova s velikom mogućnošću prilagodbe raznim željama i potrebama.
Sve zajedno implementirano je kroz sustav \emph{interaktivne bilježnice} Jupyter notebook\footnote{\url{https://jupyter.org/}} koja omogućuje brzu i jednostavnu obradu podataka kao i njeno dijeljenje sa svim zainteresiranim stranama.
Jupyter notebook je web aplikacija otvorenog koda koja se može izvršavati na lokalnom računalu ili koristeći resurse računalstva u oblaku.
Podržava razne programske jezike poput Julia, Ruby, R, C++ i mnoge druge te u ovom radu korišten Python.
Jupyter notebook omogućuje kreiranje i dijeljenje dokumenata koji sadrže izvršivi programski kod, jednadžbe, grafove i vizualizacije te popratni tekst u jednoj cijelini koju trenutno drugim načinima nije moguće ili je vrlo kompleksno za postići.
Područja primjene su najčešće obrada i transformacija podataka, numeričke analize, statistički modeli, vizualizacija podataka, strojno učenje i još mnogo toga.

U alatima tipa Excel, gdje su korištene formule skrivene iza podataka u ćelijama te se greške lako potkradu i još lakše ostanu nezamječene. 
Istraživanja su pokazala značajnu količinu grešaka u Excel proračunskim tablicama koje su u dnevnoj uporabi diljem organizacija, od kojih neke su imale i značajne negativne financijke implikacije\cite{panko1998we}. 
Iako je istraživanje starijeg datuma jedan od glavnih razloga pogrešaka (skrivene formule) je i dalje prisutan te je realno za očekivati kako se pogreške i dalje događaju, a sve većom uporabom proračunskih tablica za očekivati je kako broj istih s greškama raste.

Primjenom interaktivnih alata poput Jupyter notebooka koji imaju vidljivo prikazane formule koje koriste za proračun i kod za manipulaciju podataka, lakše se mogu uočiti pogreške unutar istih te se mogu ispraviti.
Primjenom metodologije testiranja koja je poznata u industriji razvoja softwarea, pogreške se mogu dodatno smanjiti.
Jedan od poznatijih projekata analize velike količine podataka je zasigurno detekcija gravitacijskih  valova nastalih spajanjem dvaju crnih rupa koristeći LIGO teleskop (\textit{Laser Interferometer Gravitational-Wave Observatory})\footnote{\url{https://www.gw-openscience.org/tutorials/}}

\section{NumPy}
NumPy je fundamentalni paket za numeričku analizu koristeći Python.
Između ostalih funkcija sadrži
\begin{itemize}
\item snažan alat za rad s N-dimenzionalnim poljima
\item alat za intergraciju C/C++ i Fortran programskog koda
\item korisne alate za algebarske operacije, Fourierovu analizu i ostale numeričke mogućnosti
\end{itemize}

Osim što značajno olakšava numeričku i statističku analizu skupa podataka, NumPy zbog svoje strukture i reprezentacije polja podataka omogućava značajno poboljšanje performansi obrade podataka.
Kako bi demonstrirali razliku u performansama između čistog Pythona i NumPy biblioteke možemo napraviti jednostavan eksperiment koji se sastoji od sumiranja elemenata u polju veličine $10^6$ elemenata i mjeriti vrijeme potrebnog za izvršenje zadatka.

\begin{figure}[!h]\begin{center}
\includegraphics[width=1\textwidth]{{"../resources/npSpeedTest"}.png}
\caption{Usporedba performanski Pythona i NumPy biblioteke}\label{data:npSpeedTest}
\end{center}\end{figure}
Na slici \ref{data:npSpeedTest} prikazana je razlika u brzini izvršavaja operacija sumiranja zadanog polja elemenata.
Vrijeme potrebno za sumiranje elemenata koristeći samo Python iznosi $26.8 ms \pm 750 \mu s$ dok koristeći NumPy vrijeme za isti zadatak iznosi $403\mu s \pm 2.46\mu s$\footnote{Rezultati mogu varirati u zavisnosti o korištenom računalu}.
Vidljiva je značajna razlika u potrebnom vremenu za izvršenje zadatka, a iskustva industrije\cite{van2011numpy} pokazuju još veću razliku pri kompleksnijim zadatcima.

\section{Git}
Git\footnote{\href{https://git-scm.com/}{https://git-scm.com/}} je distribuirani sustav za verzioniranje koda i ostalog rada kojeg želimo dijeliti sa suradnicima.
Git svojim jednostavnim i brzim granama omogućuje lakši razvoj proizvoda kao i ispitivanje mogućnosti i funkcija.
Kada se želi ispitati neka funkcionalnost bez ugrožavanja dosadašnjeg rada nema potrebe kopirati cijeli projekt u novi folder i onda u njemu testirati već se jednostavno kreira nova grana u kojoj se radi razvoj i kada smo sigurni da sve radi kako želimo onda se ta grana ujedini s glavnom granom projekta koja prihvati dodatne funkcionalnosti razvijene za proizvod.
Kako je Git lagan za resurse može se kreirati vrlo veliki broj grana za razne potrebe bez značajnog utjecaja na performanse razvojnog računala ili potrošnje spremišnog prostora.

S obzirom na distibuiranu narav Gita, svaki suradnik koji radi na projektu ima svoju kopiju na kojoj radi te nije vezan za neki server i stalnu komunikaciju s ostatkom tima, već je ista potrebna samo kada se povlače i šalju učinjene promjene.

Git je nastao 2005 godine za potrebe razvoja Linux jezgre i od tada je poprimio mnoge simpatije unutar inženjerske zajednice koja ga koristi kako bi zajednički razvijala projekte.

Kako bi se olakšalo dijeljenje i suradnja na projektima, 2008. godine je pokrenut GitHub - centralno mjesto za usluge poslužitelja\footnote{\href{https://github.com/features}{https://github.com/features}} (\textit{hosting}) putem kojeg je moguće pratiti životni ciklus i povijest projekta.
Svatko može pronaći projekt koji ga zanima te, ukoliko ima dovoljno vremena i znanja, može pridonjeti njegovom razvoju.
Brojne kompanije koriste GitHub kako bi podijelile svoje projekte. Podatak od travnja 2019. godine kaže kako više od 2.1 miljuna kompanija i organizacija koristi GitHub.
Jedna od njih je i Adafruit - kompanija koja proizvodi elektroničke dodatke za Arduino i druge platforme i fokusirana je na edukaciju posebice mladih (i onih koji se tako osjećaju), a njihov GitHub sadrži više od 1100 repozitorija\footnote{\href{https://github.com/adafruit}{https://github.com/adafruit/}}.
Upravo je njihov GPS Logger Shield korišten u ovom projektu, a dostupnost dokumentacije i podrška jedan je od glavnih razloga zašto je odlučeno korištenje upravo ovog proizvoda.

\chapter{Prikupljanje podataka - hardware}
\section{GPS logging shield}

\begin{figure}
\includegraphics[width=1\textwidth]{{"../tempSensor/TMP36_schematic"}.jpg}
\caption{Shema spoja TMP36 senzora}\label{shema:TMP36}
\end{figure}

\begin{figure}
\includegraphics[angle=90,width=1.1\textwidth]{{"../Adafruit-GPS-Logger-Shield-PCB/Adafruit GPS Logger Shield"}.pdf}
\caption{Shema Adafruit GPS Logger Shield}\label{shema:AdafruitLoggerShield}
\end{figure}

Na shemi \ref{shema:AdafruitLoggerShield} nalazi se shema gotovog elektroničkog sklopa kako dolazi iz tvornice\footnote{Kompletna dokumentacija dostupna je na \url{https://learn.adafruit.com/adafruit-data-logger-shield?view=all}}\footnote{GitHub repozitorij korištene verzije dostupan na \url{https://github.com/adafruit/Adafruit-GPS-Logger-Shield-PCB}}.
Na samoj tiskanoj pločici postoji tkz. prototipno područje za dodavanje vanjskih elemanata čiji je raster 2.54 mm koji odgovara standardu \textit{true-hole} elemenata.
Na to područje je dodan temperaturni senzor TMP36\footnote{Datasheet dostupan na \url{https://github.com/KristijanCetina/BachelorThesis/blob/master/resources/TMP35_36_37.pdf}}\label{ftn:TMP36Datasheet} zajedno s dodatnim pasivnim elementima koji služe kao filter smetnji koje se javljaju u radu zbog okoline.
Shema spoja je prikazana na slici \ref{shema:TMP36}.

\section{Prikupljanje podataka o temperaturi}
Kako svaki elektronički sklop ima definirani raspon radne temperature bitno je znati u kojim uvjetima isti se nalazi.
Ukoliko je temperatura previsoka može se uključiti aktivno hlađenje ili ako se unaprijed zna da će se sklop nalaziti pod povišenom radnom temperaturom onda se može konstrurati adekvatan sustav hlađenja.
Isto vrijedi za prenisku temperaturu.
Prema ranije spomenutoj shemi \ref{shema:TMP36} dodatn je temperaturni senzor koji mjeri radnu temperaturu okoline uređaja.
Pri testiranju ova vrsta senzora se pokazala veoma pouzdana, uz minimalno samozagrijavanje koje bi utjecalo na točnost mjerene veličine ali je isto tako pokazala vrlo brze promjene izlazne vrijednosti koja može biti do vanjskih smetnji.
Kako bi se otklonio taj problem primjenjena su dva rješenja.
Prvi je hardwareski fiter - kondenzatori koji je prikazan na shemi \ref{shema:TMP36}, a drugi je softwareski fiter.
Tvornički podaci o izlaznom naponu šuma mogu se pronaći u datasheetu uređaja, slika 20.
Na slici \ref{data:RawWithOutCaps} prikazane su izlazne vrijednosti senzora bez ikakvog filtriranja i obrade.
frekvencija uzorkovanja je 10Hz (10 očitanja u sekundi)
Svakako nije realno za očekivati da se temperatura mjenja sukladno očitanim vrijednostima.

\subsection{Softwareski filter}
Softwareski filter radi na principu da očitava 10 vrijedosti sa senzora te ih sprema u polje.
Potom ih sortira po veličini i uzima medijan\footnote{Medijan (mediana, centralna vrijednost) je pojam iz statistike koji određuje sredinu distribucije. Pola vrijednosti skupa (distribucije) nalazi se iznad mediane, a pola ispod} vrijednost kao točnu temperaturu.
Na taj način se postiže da se eliminiraju sve vrlo visoke i vrlo niske vrijednosti koje se mogu pojaviti zbog šuma u signalu.
Vrijenosti se čitaju svakih 100 ms te uz račuanje na bazi 10 vrijednosti daje frekvenciju od 1 očitanja u sekundi koja odgovara i frekvenciji uzorkovanja podataka sa GPS senzora.
Prilikom testiranja utvrđeno je da veći broj uzoraka ne doprinosti kvaliteti izmjerenih vrijednosti, a pro manjem broju uzoraka može se potkrasti poneka nerealna vrijenosti.
Kako je očekivano vrijeme promjene temperature značajno duže od 1 sekune onda su prihvaćene navedene vrijednosti i metoda filtriranja.
Na slici \ref{data:FilteredWithOutCaps} prikazane su izlazne vrijendosti senzora nakon primjene opisanog softwareskog filtera.
Primjeti se značajno manje skokova od nečega što se može smatrati stvarna vrijendost. 


%\begin{figure}
%\includegraphics[width=0.5\textwidth]{{"../tempSensor/RawWithOutCaps"}.jpg}
%\caption{Vrijednosti senzora bez filtriranja}\label{data:RawWithOutCaps}
%\end{figure}

\begin{figure}
\begin{minipage}[c]{0.49\linewidth}
\includegraphics[width=\linewidth]{{"../tempSensor/RawWithOutCaps"}.jpg}
\caption{Vrijednosti senzora bez \mbox{filtriranja}}\label{data:RawWithOutCaps}
\end{minipage}
\hfill
\begin{minipage}[c]{0.49\linewidth}
\includegraphics[width=\linewidth]{{"../tempSensor/FilteredWithOutCaps"}.jpg}
\caption{Vrijednosti senzora sa \mbox{softwareskim} filtriranja}\label{data:FilteredWithOutCaps}
\end{minipage}%
\end{figure}

\subsection{Hardwareski filter}
Filter je jednostavna mreža keramičkih kondenzatora vrijednosti $10 pF$ koji su spojeni što bliže senzoru između izvoda za napajanje i izlaza senzora prema točki nultog potencijala (\textit{GND, masa}) kako bi apsorbirali eventualne smetnje.
Iako je softwareski fiter u nekim situacijama dovoljno dobar ovo jednostavno i jeftino rješenje daje dodatan sloj filtriranja koji za posljedicu ima vrlo glatko očitanje temperature bez skokova u vrijednostima.


\chapter{Obrada podataka - software}
U ovom poglavlju biti će opisana softwareska komponenta za obradu i vizualizaciju prikupljenih podataka.





\chapter{Zaključak}\label{ch:Zakljucak}
U ovom radu prikazan je postupak izrade uređaja za prikupljanje podataka kao i analiza prikupljenih podataka.
Sve to je urađeno primjenom komponenti s otvorenom i besplatnom dokumentacijom te softwarea koji je isto tako slobodan i besplatan za korištenje.
Time je potvrđena hipoteza iznesena u uvodu ovog rada.

Sve ovo je moguće zahvaljujući popularnosti koji su korišteni sustavi postigli zbog svojeg \textit{open soucre} pristupa radu te su zbog svoje raširenosti i prihvaćenosti omogućili razvoj zajenice koja podupire takav pristup.
Upravo je ta zajednica autor mnogih vodiča, knjiga, kratki demo primjer i ostalih dragocjenih resursa koji omogućavaju savladavanje temetike i ulazak u svijet elektronike i razvoja softwarea i relativnim početnicima koji imaju volje za učenje.

\nocite{*}
\addcontentsline{toc}{chapter}{Literatura}
\bibliography{literatura}

%Ako treba dodati kod kao appendix 
\newpage
\appendix
\begin{appendices}\appendix
\chapter{Programski kod na Arduino mikroračunalu}\label{ArduinoSource}
\includepdf[fitpaper, pages=-]{../arduinoSource/LoggingWithTemp/arduinoSource.pdf}
\chapter{Analiza podataka kretanja vozila}\label{AnalizaKretanjaVozila}
\includepdf[fitpaper, pages=-]{../dataAnalysis/AnalizaKretanjaVozila.pdf}
\chapter{Vizualizacija podataka kretanja lifta}\label{VizualizacijaKretanjaLifta}
\includepdf[fitpaper, pages=-]{../dataAnalysis/KretanjeLifta.pdf}
\end{appendices}

\end{document}