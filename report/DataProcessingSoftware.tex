\chapter{Obrada podataka - software}\label{ch:Software}
Kako je već spomenuto u uvodu i opisu korištenih tehnologija, analiza podataka izrađena je koristeći Python programski jezik unutar Jupyter Notebook interaktivne bilježnice uz dodatak Numpy biblioteke za numeričku analizu i Matplotlib biblioteke za izradu grafova.

\section{Prikaz podataka vozila u kretanju}
Kao prvi primjer obrade podataka izrađena je analiza kretanja vozila i vizualizirani su prikupljeni podatci.
U dodatku \ref{AnalizaKretanjaVozila} prikazana je cjelokupna analiza podataka te demonstriran završni proizvod koji može biti dalje distribuiran u printanoj verziji ili preferirano u digitalnom obliku koji onda omogućava daljni rad na istome.

Koristeći razne tipove ćelija unutar Jupyter notebooka, stvorena je prikazana analiza.
Tip čelije \textit{Markdown} podržava formatiranje i sintaksu \texttt{.md} Markdown datoteke\footnote{\url{https://www.markdownguide.org/}}, a koje se koriste za opisni dio analize kako bi čitatelj znao o čemu se radi bez potrebe proučavanja programskog koda.

Tip ćelije \textit{Code} omogućuje unošenje i izvršavanje programskoga koda koji će napraviti neku željenu radnju.
Ćelije se mogu izvršavati jedno po jedna ili sve u sekvencijalnom nizu.

Na kraju postoje i \textit{Output} ćelije koje prikazuju izlazni rezultat izvšene \textit{code} ćelije.

Kako bi mogli koristiti navedene dodatne module u Python prvo ih je potrebno uvesti
\begin{lstlisting}[language=Python]
import pandas as pd
import matplotlib.pyplot as plt
from numpy import genfromtxt, arange, sin, pi
from matplotlib import style
from matplotlib import dates as mpl_dates
import numpy as np
from pandas.plotting import register_matplotlib_converters
register_matplotlib_converters()
\end{lstlisting}

Konvencija je da se NumPy biblioteka skrećeno imenuje \texttt{np} kako bi se olakšalo kasnije korištenje u radu.
Isto vrijedi i za ostale često korištene biblioteke.

Nakon toga potrebno je navesti koja se datoteka s podatcima koristi i nazvati kolone podataka 
\begin{lstlisting}[language=Python]
filename='GPSLOG10.CSV'
data=pd.read_csv(filename, header=None, delimiter=',',
names=['Sentence','Time','Validity','Latitue','NS','Longitude','EW',
'Speed','Direction','Date','NA1','NA2','Checksum','Temperature'])
\end{lstlisting}
Pri čemu je \texttt{GPSLOG10.CSV} ime datoteke u kojem su spremljeni podatci.

Nakon uvodnih radnji, spremni smo započeti s analizom podataka i vizualizcijom onih koje nas zanimaju.
U ovisnosti o željenim rezultatima ovisit će i potrebne operacije koje je potrebno poduzeti.
Po nekad su podatci već u takvom formatu da se mogu odmah koristiti, dok je ponekada potrebno napraviti veću ili manju manipulaciju nad njima kako bi bili pogodni za korištenje.
U svakom slučaju, vrlo je važno poznavati strukturu seta podataka koji se koristi kako bi se izbjegle pogreške zbog njegovog nepoznavanja ili korištenja krive mjerne jedinice.
Godine 1999., NASA je izgubila 125 milijuna vrijedan \textit{Mars Climate Orbiter} zbog previda pretvorbe jedinica\cite{sauser2009projects} što samo pokazuje koliko je to realna zamka u koju i najbolji svjetski stručnjaci mogu upasti.
U ovom slučaju, brzina je prikazana u čvorovima što odgovara prijeđenom putu od jedne nautičke milje u sat vremena.
Ukoliko želimo prikazati brzinu u $km/h$ potrebno je izvršiti pretvorbu jedinica.
Jedna nautička milja iznosi 1852 $m$ što znači da zapisanu brzinu treba pomnožiti s 1.852 kako bi dobili $km/s$.

\begin{figure}[!h]\begin{center}
\includegraphics[width=1\textwidth]{{"../dataAnalysis/GrafKretanjaBrzineVozila"}.png}
\caption{Izrađeni graf kretanja vozila}\label{graf:Speed}
\end{center}\end{figure}

Obzirom da je uvijek korisno vizualizirati kretanje veličine na grafu tako je i u ovom radu napravljen graf promjene brzine u vremenu prikazan na \ref{graf:Speed}, što je koristeći Matplotlib vrlo jednostavno.
\begin{lstlisting}[language=Python]
plt.plot(data['Time'],data['Speed']*1.852, 'b-')
\end{lstlisting}
Pri čemu  \textit{data['Time']} predstavlja apcisu grafa, a \textit{data['Speed']} predstavlja ordinatu prikazanog grafa.
\textit{'b-'} je parametar kojim je definirana linija - linija plave boje.
Svaki graf treba biti propisno obilježen pa su tako i na ovom grafu označene vrijednosti i imena osi, prikazane mjerne jedinice i dodana legenda, iako je prikazana samo jedna veličina.

Dodatno, iz dostupnih podataka može se napraviti graf kretanja temperature koji je prikazan u kompletnoj analizi u dodatku \ref{AnalizaKretanjaVozila}.
Osim podataka koji se mogu prikazati grafički, možemo pogledati i ostale statističke podatke kao što su minimalna i maksimalna temperatura tijekom promatranog razdoblja.
\begin{lstlisting}[language=Python]
print ('Minimalna temperatura: ',np.min(data['Temperature']), 'C')
print ('Maximalna temperatura: ',np.max(data['Temperature']), 'C')
\end{lstlisting}
Ovdje su korištene ugrađene funkcije \texttt{min} i \texttt{max} NumPy biblioteke koje za parametar uzimaju set podataka temperature.

Zanima li nas prosječna temperatura tokom promatranog razdoblja, tada i za to postoji urađena funkcija koja nam olakšava rad.
Umjesto da zbrajamo sve elemente te zbroj djelimo s brojem elemenata, jednostavno možemo koristiti
\begin{lstlisting}[language=Python]
print ('Prosjecna temperatura: ', '
{:.2f}'.format(np.mean(data['Temperature'])), 'C ',
'{:.2f}'.format(np.std(data['Temperature'])), 'C')
\end{lstlisting}
Pri čemu \texttt{np.mean} daje prosječnu temperaturu, a \texttt{np.std} daje standardnu devijaciju vrijednosti.
\texttt{'{:.2f}'.format} na opcija formatiranja stringa pomoću koje se prikazuje dva decimalna mjesta.

\section{Prikaz podataka lifta}
Analiza prikazana u dodatku \ref{VizualizacijaKretanjaLifta} vizualizira podatke prikupljene mobilnim telefonom koristeći aplikaciju \textit{phyphox}\footnote{\url{https://www.phyphox.org}}.
Korišten je modul koji očitava podatke s akcelerometra u sve 3 osi koordinatnog sustava.
Mobilni uređaj bio je prislonjen na bočni panel lifta i pokušano je postaviti ga da stoj okomito na smjer kretanja sa stražnojm stranom priljubljenim na panel kako bi se dobili čim točniji podatci bez unošenja pogreške zbog pomaknutih osi mobitela i lifta.
Izmjerene vrijednosti predstavljaju put od prizemlja do 5. kata zgrade na adresi Koparska 58, Pula.

Podatci su zatim spremljeni u .csv datoteku te je ista prenesena na računalo.
Datoteka je formatirana na način da ima imena vrijednosti u zaglavlju te su korišteni zadani nazivi, a polja odvojena znakom ";".