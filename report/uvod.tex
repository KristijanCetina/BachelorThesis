\chapter{Uvod i opis zadatka}\label{OpisIOgranicenja}
Tema iz ovog rada nastala je iz autorove želje za proučavanjem tematike te kao gorljivi poklonik medote učenja kroz praktičan rad i primjenu stečenog znanja i iskustva na rješavanje realnog problema nastao je ovaj rad.
Dodatno, autorova želja je implementirati stečena znanja i razvijeni hardware i software u budućim projektima koje ima na umu.
Koristeči dostupna i otvorena znanja značajno je lakše razvijati potrebne projekte, a otvarajući rad ostalim zainteresiranom stranama omogućava se kolaboracija na usavršavanju projekta ili stvaranju baze za druge, kompeksnije projekte.

Arduino platforma je odabrana za korištenje u hardwareskom djelu rada jer ista predstavlja relativno jednostavan i povoljan pristup te uz veliki izbor dodatnih gotovih pločica za proširivanje mogućnosti (\textit{shields}) s otvorenom dokumentacijom logičan je izbor.

Na softwareskoj strani izbor je pao na programski jezik Python te dodatne biblioteke Matplotlib i NumPy koje su isto tako otvorenog koda i besplatne za korištenje i dovoljno jednostavne za učenje.
Dodajmo na to popularnost među akademskom i znanstvenom zajednicom isti se nameću kao logičan izbor.

\section{Opis i definicija problema}
Pri proučavanju bilo kojeg inženjerskog ili znanstvenog problema potrebno je prikupiti dovoljnu količinu podataka te iste analizirati i na reprezentativan način prikazati u pokušaju razumjevanja problema.
GPS je glogalno dostupan sustav putem kojeg se mogu prikupiti dovoljno precizni podaci za civilnu primjenu te je format podataka standardiziran te se zbog toga mogu koristiti i standardne metode analize podataka.
Ovisno o promatranom problemu potrebni su razni dodatni podaci koje pružaju razni senzori.
Temperatura je jedan o njih te je u ovom radu korišten klasični TMP36 temperaturni senzor i obrađeni podaci koji su skupljeni putem navedenog senzora.

\section{Cilj i svrha rada}
Cilj ovog rada je bio izraditi jednostavan snimač podataka (\textit{datalogger}) koji će spremati GPS podatke zajedno s podacima prikupljenim sa instaliranih senzora za kasniju analizu.
Izrađeni uređaj je namjenjen kao snimač podataka u kompleksnijem sklopu koji se može koristiti kada god postoji potreba za loggiranje podataka.
Uređaj je namjenjen da zadovolji široki spektar potreba koje se mogu javiti bilo u industriji npr. prilikom praćenja pošiljki ili pak prilikom skupljanja podataka u istraživačke svrhe kako bi se razumio širi problem.

Sklop je baziran na Arduino platformi koja omogućava lak razvoj prototipova uz široku dostupnost gotovih dodatnih modula (\textit{shields}) koji se jednostavno spajaju na bazno mikroračunalo.

Prikpljeni podaci se spremaju na SD karticu na uređaju u datoteku za kasniju obradu i analazu.
Prikupljeni podaci se uz pomoć programskog jezika Python i dodatnih modula za statističku i numeričku analizu kao što su Pandas i Matplotlib obrađuju kroz sučelje interaktivne bilježnice Jupyter Notebook.
Pristup obrade putem interaktivne bilježnice uz korištenje raznih tipova čelija kao što su \textit{Code Cells, Markdown Cells i Raw Cells} omogućava lakšu vizualizaciju i pregled samog rada koji je pogodan za kasnije dijeljenje svim zainteresiranim stranama koji žele pregledati ili nastaviti rad na analizi.

\section{Hipoteza rada}
Hipoteza ovog rada je da se primjenom pristupa otvorenog i slobodnog koda (računanog koda, metode obrade podataka, algoritama i shema hardwareskog sklopa) može izraditi funkcionalan sustav uz prihvatljive troškove i skromne resurse.

\section{Metode rada}
Tijekom izrade ovog rada korištene su različite znanstveno-istraživačke metode od kojih je svaka najprikladnija postavljenom izazovu, a one su:
\begin{itemize}
\item Istraživačka metoda – za stjecanje uvida u zadane okvire zadatka
\item Metoda logičke analize i sinteze – za prikupljanje podataka iz literature
\item Deskriptivna metoda – za izradu uvodnog i završnog djela projektnog zadatka
\item Eksperimentalna metoda - u potrazi za optimalnim rješenjima za zadani dio problema
\end{itemize}

\section{Struktura rada}
Struktura ovog rada podjeljena je u logičke cjeline.
Nakon uvoda i objašanjenja rada u poglavlju \ref{technologyStack} opisana je korišten pristup i primjenjene tehnologije kao i dano objašnjenje zašto je ista upotrebljena.

U poglavlju \ref{ch:Hardware} opisan je izrađen hardwareski elektronički sklop korišten za prikupljanje obrađenih podataka.
U dodatku \ref{ArduinoSource} priložen je izvorni kod koji se izvršava na Arduino mikroračunalu.

Poglavlje \ref{ch:Software} opisuje postupak izrade analize prikupljenih podataka.
Kompletna analiza se nalazi u dodatku \ref{AnalizaKretanjaVozila}

Kompletan Git repozitorij ovog rada javno je dostupan na \url{https://github.com/KristijanCetina/BachelorThesis}