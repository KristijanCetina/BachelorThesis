\chapter{Uvod i opis zadatka}\label{OpisIOgranicenja}

\section{Opis i definicija problema}

\section{Cilj i svrha rada}
Cilj ovog rada je bio izraditi jednostavan snimač podataka (\textit{datalogger}) koji će spremati GPS podatke zajedno s podacima prikupljenim sa instaliranih senzora za kasniju analizu.
Izrađeni uređaj je namjenjen kao snimač podataka u kompleksnijem sklopu koji je koji se može koristiti kda god postoji potreba za loggiranje podataka.
Uređaj je namjenjen da zadovolji široki spektar potreba koje se mogu javiti bilo u industriji npr. prilikom praćenja pošiljki ili pak prilikom skupljanja podataka u istraživačke svrhe kako bi se razumio širi problem.

Sklop je baziran na Arduino platformi koja omogućava lak razvoj prototipova uz široku dostupnost gotovih dodatnih modula (\textit{shields}) koji se jednostavno spajaju na bazno mikroračunalo.

Prikpljeni podaci se spremaju na SD karticu na uređaju u datoteku za kasniju obradu i analazu.
Prikupljeni podaci se uz pomoć programskog jezika Python i dodatnih modula za statističku i numeričku analizu kao što su Pandas i Matplotlib obrađuju kroz sučelje interaktivne bilježnice Jupyter Notebook.
Pristup obrade putem interaktivne bilježnice uz korištenje raznih tipova čelija kao što su \textit{Code Cells, Markdown Cells i Raw Cells} omogućava lakšu vizualizaciju i pregled samog rada koji je pogodan za kasnije dijeljenje svim zainteresiranim stranama koji žele pregledati ili nastaviti rad na analizi.

\section{Hipoteza rada}

\section{Metode rada}

\section{Struktura rada}
Kompletan Git repozitorij ovog rada javno je dostupan na \url{https://github.com/KristijanCetina/BachelorThesis}