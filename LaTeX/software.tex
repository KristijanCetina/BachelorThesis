\chapter{Software}
U ovom pogljaću biti će opisana softwareska komponenta za obradu i vizualizaciju prikupljenih podataka.
\section{Prikupljanja podataka: software na Arduinu}
Sustav skuplja i sprema podatke sa GPS receivera kao i s temperaturnog senzora koje sprema u .csv \footnote{Comma Separated Values} formatu koji je pogodan za kasniju obradu bilo putem Excel programskog alata ili drugih alata za obradu i vizualizaciju podataka.


\section{Obrada i prezentacija prikupljenih podataka: Jupyter Notebook}
U ovom radu za obradu i prikazivanje podataka korišten je programski jezik Python\footnote{\href{https://www.python.org/}{https://www.python.org/}} uz dodatke Pandas\footnote{\href{https://pandas.pydata.org/}{https://pandas.pydata.org/}} i Matplotlib\footnote{\href{https://matplotlib.org/}{https://matplotlib.org/}}.
Pandas omogućava lakšu manipulaciju podacima dok Matplotlib omogućava izradu kvaitetnih grafova s velikom mogučnošću prilagodbe raznim željama i potrebama.
Sve zajedno je implementirano kroz sustav \emph{interaktivne bilježnice} koja omogućava brzu i jednostavnu obradu podataka kao i njeno dijeljenje sa svim zainteresiranim stranama.