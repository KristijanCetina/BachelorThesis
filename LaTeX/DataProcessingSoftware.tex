\chapter{Obrada podataka - software}
Kako je već spomenuto u uvodi i opisu korištenih tehnologija analiza podataka izrađena je koristeći Python programski jezik unutar Jupyter Notebook interaktivne bilježnice uz dodatak Numpy biblioteke za numeričku analizu i Matplotlib biblioteke za izradu grafova.

Kao prvi primjer obrade podataka izrađena je analiza kretanja vozila i vizualizirani prikupljeni podaci.
U dodatku \ref{AnalizaKretanjaVozila} prikazana je kompletna analiza podataka te demonstriran završni proizvod koji može biti dalje distribuiran bilo u printanoj verziji ili preferirano u digitalnom obliku koji onda omogućava daljni njad na istome.

Koristeći razne tipove čelija unutar Jupyter notebooka stvorena je prikazana analiza.
Tip čelije \textit{Markdown} podržava formatiranje i snitaksu \texttt{.md} Markdown datoteke\footnote{\url{https://www.markdownguide.org/}}, a one se koriste za opsisni dio analize kako bi čitatelj znao o čemu se radi bez potrebe za proučavanjem programskog koda.

Tip čelije \textit{Code} omogućava unošenje i ozvršavanje programskog koda koji će napraviti neku željenu radnju.
Čelije se ogu izvršavati jedno po jedna ili sve čelije u sekvencijalnom nizu.

Na kraju postoje i \textit{Output} čelije koje prikazuju izlazni rezultat izvšrene \textit{code} čelije.

Kako bi mogli koristiti navedene dodatne module u Python ih je prvo potrebno uvesti
\begin{lstlisting}[language=Python]
import pandas as pd
import matplotlib.pyplot as plt
from numpy import genfromtxt, arange, sin, pi
from matplotlib import style
from matplotlib import dates as mpl_dates
import numpy as np
from pandas.plotting import register_matplotlib_converters
register_matplotlib_converters()
\end{lstlisting}

Konvencija je da se NumPy biblioteka skrećeno imenuje \texttt{np} kao bi se olakšalo kasnije korištenje u radu.
Isto vrijedi za ostale često korištene biblioteke.

Nakon toga je potrebno navesti koja datoteka s podacima se koristi i nazvati kolone podataka 
\begin{lstlisting}[language=Python]
filename='GPSLOG10.CSV'
data=pd.read_csv(filename, header=None, delimiter=',',
names=['Sentence','Time','Validity','Latitue','NS','Longitude','EW',
'Speed','Direction','Date','NA1','NA2','Checksum','Temperature'])
\end{lstlisting}
Pri čemu je \texttt{GPSLOG10.CSV} ime datoteke u kojem su spremljeni podaci.

Nakon uvodnih radnji sada smo spremni započeti s analizom podataka i vizualizcijom onih koje nas zanimaju.
U ovisnosti o željenim rezultatima ovisiti će i potrebne operacije koje je potrebno poduzeti.
Nekada su podaci već u takvom formatu da se mogu odmah koristiti, dok je ponekada potrebno napraviti veću ili manju manipulaciju nad njima kako bi bili pgodni za korištenje.
U svakom slučaju vrlo je bitno poznavati strukturu seta podataka koji se koristi kako bi se izbjegle greške zbog njegovog nepoznavanja ili korištenja krive mjerne jedinice.
1999. godine NASA je izgubila 125 milijuna vrijedan \textit{Mars Climate Orbiter} zbog previda pretvorbe jedinica\cite{sauser2009projects} što samo pokazuje kako to realna zamka u koji u najbolji svjetski stručnjaci mogu upasti.
U ovom slučaju brzina je prikazana u čvorovima što odgovara prijeđenom putu od jedne nautičke milje u sat vremena.
Ukoliko želimo prikazati brzinu u $km/h$ potrebno je izvršiti pretvorbu jedinica.
Jedna nautička milja iznosi 1852 $m$ što znači da zapisanu brzinu treba pomnožiti s 1.852 kako bi dobili $km/s$.

\begin{figure}[!h]\begin{center}
\includegraphics[width=1\textwidth]{{"../dataAnalysis/GrafKretanjaBrzineVozila"}.png}
\caption{Izrađeni graf kretanja vozila}\label{graf:Speed}
\end{center}\end{figure}

Kako je uvijek korisno vizualizirati kretanje veličine na grafu tako je i u ovom radu napravljen graf promjene brzine u vremenu prikazan na \ref{graf:Speed}.
Koristeći Matplotlib je to vrlo jednostavno
\begin{lstlisting}[language=Python]
plt.plot(data['Time'],data['Speed']*1.852, 'b-')
\end{lstlisting}
Pri čemu  \textit{data['Time']} predstavlja apcisu na grafu, a \textit{data['Speed']} predstavlja ordinatu prikazanog grafa.
\textit{'b-'} je parametar kojim je definirana linija - linija plave boje.
Svaki graf treba biti propisno obilježen tako su i na ovom grafu označene vrijednsti i imena osi, prikazane mjerne jedinice i dodana legenda iako je prikazana samo jedna veličina.

Dodatno iz dostupnih podataka može se napraviti graf kretanja temperature koji je prikazan u kompletnoj analizi u dodatku \ref{AnalizaKretanjaVozila}.
Osim podataka koji se mogu prikazati grafički možemo pogledati i ostale statističke podatke kao što su minimalna i maksimalna temperatura tokom promatranog razdoblja.
\begin{lstlisting}[language=Python]
print ('Minimalna temperatura: ',np.min(data['Temperature']), 'C')
print ('Maximalna temperatura: ',np.max(data['Temperature']), 'C')
\end{lstlisting}
Ovdje su korištene ugrađene funkcije \texttt{min} i \texttt{max} NumPy biblioteke koje za parametar uzimaju set podataka temperature.

Ako nas zanima prosječna temperatura tomok promatranog razdoblja onda i za to postoji urađena funkcija koja nam olakšava rad umjesto da zbrajamo sve elemente te zbroj djelimo s brojem elemenata jednostavno možemo koristiti
\begin{lstlisting}[language=Python]
print ('Prosjecna temperatura: ', '
{:.2f}'.format(np.mean(data['Temperature'])), 'C ',
'{:.2f}'.format(np.std(data['Temperature'])), 'C')
\end{lstlisting}
Pri čemu \texttt{np.mean} daje prosječnu temperaturu, a \texttt{np.std} daje standardnu devijaciju vrijednosti.
\texttt{'{:.2f}'.format} na opcija formatiranja stringa pomoću koje se prikazuje dva decimalna mjesta.
